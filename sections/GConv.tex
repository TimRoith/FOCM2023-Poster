\newdimen\midWidth
\newdimen\midHeight
\midWidth=.31\textwidth
\midHeight=.17\textheight
% The following empty line is intentional!

% >>>>>>>>>>>>>>>>>>>>>>>>>>>>>>>>>>>>>>>
% Seperation Line
% <<<<<<<<<<<<<<<<<<<<<<<<<<<<<<<<<<<<<<<
%\vfill%
\begin{minipage}[t][\seplineHeight][b]{\textwidth}%
%\fbox{%
\vbox to \seplineHeight{%
\vfill%
\begin{center}%
\textcolor{sky}{%
\rule{\textwidth}{.2mm}}%
\end{center}%
\vfill%
}%
%}%
\end{minipage}%
\noindent%

\begin{minipage}[b][\midHeight][t]{\topWidth}%
{\color{BaseDarkColor}\usebeamerfont{block title} Graph Resolution}\\%
Using the Hausdorff distance
% 
\begin{align*}
d_H(A,B)=\sup_{x\in A} \dist(x,B) \vee \sup_{x\in B} \dist(x,A)
\end{align*}
we consider the \color{ponk}\textbf{graph resolution}\nc:
\begin{align*}
\gres_n=
d_H(\domain_n, \domain) 
\vee d_H(\constr_n, \constr).
\end{align*}
%
\nc
%
%
%
{\color{BaseDarkColor}\usebeamerfont{block title} \noindent Local Convexity}%

%
%
%
\begin{minipage}{.4\textwidth}%
\begin{tikzpicture}[scale=2.5]
\begin{scope}
\clip(1.2,1.0) rectangle (4.5,5.5);
\clip(1.0,1.0) circle (4.1cm);

\begin{scope}
%\path [scope fading=my fading,fit fading=false] (1,0) rectangle (8,8);
\path[fill=sky!30, draw = none, path fading=south] plot coordinates {(0,0)(0,8)(4,8)(2,2)(6,4)(8,0)} node at (1.8,1.7) {};
\end{scope}
\begin{scope}[every node/.style={circle,thick,sky,draw, inner sep=6pt}]
\node (X1) at (2.5,4.5) {};
\node (X2) at (2.25,3.75) {};
\node (X3) at (2.0,3.0) {};
\node (Y1) at (4.2,2.8) {};
\node (Y2) at (3.5,2.45) {};
\node (Y3) at (2.8,2.1) {};
\end{scope}
\node[circle,fill,draw,grape,inner sep=0pt] (Z) at ($(X1) - 1.03*(1,3)$) {};
%\node[circle,fill=faulnat,inner sep=0,text=sky] (J) at (1.5,5.2) {$x$};

\begin{scope}[every path/.style={grape,dotted,thick}]
\path (X1) edge (X2);
\path (X2) edge (X3);
\path (Y1) edge (Y2);
\path (Y2) edge (Y3);
\end{scope}
%\path[draw=sky] [-] (B) edge (J2);
\path[grape,thick] (X3) edge (Z);
\path[grape,thick] (Y3) edge node[sloped,below=5pt,pos=0.5] {\small\nc$d_{\domain}$} (Z);
\begin{scope}[every path/.style={sky!20,dotted, thick}]
\path (X1) edge  node[sloped, above] {\small$|x_i-y_i|$} (Y1);
\path (X2) edge (Y2);
\path (X3) edge (Y3);
\end{scope}
\end{scope}
\end{tikzpicture}
\end{minipage}%
%
%
\begin{minipage}{.6\textwidth}%
We assume that $\domain$ is \alert{locally convex}, i.e., $\forall x,y\in\domain$:
\begin{align*}
d_\domain(x,y) \leq \abs{x-y} + \phi(\abs{x-y})
\end{align*}
where ${\phi(\gscale)}\ll{\gscale}$ as $\gscale\to 0$.

$\Rightarrow$ No sharp internal corners.
\end{minipage}%
\end{minipage}%
%
%
%
\hfill%
%
%
%
%\fbox{%
\begin{minipage}[b][\midHeight][t]{\topWidth}%
\tikzfading[name=fade out, 
    inner color=transparent!0,
    outer color=transparent!60]
\hspace*{2em}\raisebox{-17em}{%
\begin{tikzpicture}[scale=2.7]
\draw[fill=sky!10, draw = none, thick, dashed] plot [smooth cycle] coordinates {(-1,.1)(-0.2,5.5)(4,4.9)(4.5,3)(6,1)(4,-0.5)(2,-1)} node at (1.8,1.7) {};
%
\node (F) at (3.5,1.5) {};
\fill[apple!20,path fading = fade out] (F) circle (57pt);
%
\begin{scope}[every node/.style={circle,very thick,sky,draw,inner sep=6pt}]
\node (A) at (0.7,1.9) {};
\node (B) at (1.8,1.3) {};	
\node[grape] (C) at (2.,5.2) {};
\node[grape] (D) at (-0.7,2.5) {};
\node[] (F) at (F) {};
\node[grape] (G) at (-0.2,1.5) {} ;
\node[grape] (E) at (0.2,3.8) {};
\node (H) at (2.8,4) {};
\node (I) at (2.5,3.) {};
\end{scope}

\draw[->,>=Latex,apple, thick] (F) -- 
    +(-135:57pt) node[midway,below,sloped]{\small\nc$\sim \scaling_n$};

\node (Omega) at (0,5) {$\domain$};
\node (Omegan) at (3.5,4.2) {$\domain_n$};
\node (On) at ($(E)+(0.5,0.5)$) {$\mathbin{\constr_n}$};

% boundary
\begin{scope}
\clip (-0.5,-1.2) rectangle (3,6);
\draw[fill=none, draw = grape, very thick, dashed] plot [smooth cycle] coordinates {(-1,.1)(-0.2,5.5)(4,4.9)(4.5,3)(6,1)(4,-0.5)(2,-1)} node at (2.2,5.7) {$\mathbin{\constr}$};
\end{scope}
\draw[grape,very thick,dashed] plot [smooth] coordinates {(-0.5,-0.44)(-.7,.2)(0,3.5)(1.5,2.5)(2,3.0)};
%
\begin{scope}[every path/.style={sky}]
\path [-] (A) edge (B);
\path [-] (A) edge (E);
\path [-] (A) edge (D);
\path [-] (A) edge (G);
\path [-] (D) edge (G);
\path [-] (C) edge (H);
\path [-] (D) edge (E);
\path [-] (B) edge (F);
\path [-] (I) edge (F);
\path [-] (I) edge (B);
\path [-] (I) edge (H);
\end{scope}
%
\node[circle,very thick,ponk,draw, fill, inner sep=3pt] (J) at (0.5,-0.5) {};
\path[draw=ponk, dotted, very thick] [-] (B) edge node[midway,below, sloped] {\small$\gres_n$} (J) ;
\end{tikzpicture}}%
%
\end{minipage}%
%}%
%
%
%
%
%
%
\hfill%
%
\begin{minipage}[b][\midHeight][t]{\topWidth}%
{\color{BaseDarkColor}\usebeamerfont{block title} $\Gamma$-Convergence}\\%
%
Considering the functionals 
%
\begin{align*}
E_\infty^{w_n}(\vec u) &:= \frac{1}{\scaling_n}\ \vec E_\infty^{w_n}(\vec u)  &&+ \color{grape} \textbf{constraint},\\
\func_\infty(u) &:= \norm{\nabla u}_{L^\infty(\domain)}
&&+ \color{grape} \textbf{constraint},
\end{align*}
%
%
we prove in \cite{roith2022continuum}:
%
\begin{itemize}
\item $E_\infty^{w_n}\xrightarrow[]{\hspace*{.5cm}\Gamma\hspace*{.5cm}}
\sigma_\eta~\func_\infty$ in a suitable topology,
\item $\sup_{n\in\N}E_\infty^{w_n}(\vec u_n)<\infty$ and boundedness imply that $(\vec u_n)_{n\in\N}$ is relatively compact.
\end{itemize}
%
Here we assume the \alert{weakest scaling}%, i.e., $\color{apple}\scaling_n\nc$ is a null sequence with 
\begin{align*}
%\frac{\color{ponk}\gres_n}{\color{apple}\scaling_n\nc}\longrightarrow 0.
\gres_n\ll\scaling_n\ll 1.
\end{align*}
\end{minipage}%
% The following empty line is intentional!

% >>>>>>>>>>>>>>>>>>>>>>>>>>>>>>>>>>>>>>>>>>>>>>>>>>>>>>>>>>>
% Seperation Line
% <<<<<<<<<<<<<<<<<<<<<<<<<<<<<<<<<<<<<<<<<<<<<<<<<<<<<<<<<<<
%\vfill%
\begin{minipage}[t][\seplineHeight][b]{\textwidth}%
%\fbox{%
\vbox to \seplineHeight{%
\vfill%
\begin{center}%
\textcolor{sky}{%
\rule{\textwidth}{.2mm}}%
\end{center}%
\vfill%
}%
%}%
\end{minipage}%